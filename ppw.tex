\documentclass[12pt,oneside]{article}
% Change "article" to "report" to get rid of page number on title page
\usepackage{amsmath,amsfonts,amsthm,amssymb}
\usepackage{setspace}
\usepackage{fancyhdr}
\usepackage{lastpage}
\usepackage{extramarks}
\usepackage{chngpage}
\usepackage{soul}
\usepackage[usenames,dvipsnames]{color}
\usepackage{graphicx,float,wrapfig}
\usepackage{ifthen}
\usepackage{listings}
\usepackage{courier}
\setlength{\headheight}{15pt}
\definecolor{MyDarkGreen}{rgb}{0.0,0.4,0.0}
\newtheorem{claim}{Claim}
% For faster processing, load Matlab syntax for listings
\lstloadlanguages{Matlab}%
\lstset{language=Matlab,
  frame=single,
  basicstyle=\small\ttfamily,
  keywordstyle=[1]\color{Blue}\bf,
  keywordstyle=[2]\color{Purple},
  keywordstyle=[3]\color{Blue}\underbar,
  identifierstyle=,
  commentstyle=\usefont{T1}{pcr}{m}{sl}\color{MyDarkGreen}\small,
  stringstyle=\color{Purple},
  showstringspaces=false,
  tabsize=5,
  % Put standard MATLAB functions not included in the default
  % language here
  morekeywords={xlim,ylim,var,alpha,factorial,poissrnd,normpdf,normcdf},
  % Put MATLAB function parameters here
  morekeywords=[2]{on, off, interp},
  % Put user defined functions here
  morekeywords=[3]{FindESS},
  morecomment=[l][\color{Blue}]{...},
  numbers=left,
  firstnumber=1,
  numberstyle=\tiny\color{Blue},
  stepnumber=5
}

% In case you need to adjust margins:
%\topmargin=-0.45in      %
%\evensidemargin=0in     %
%\oddsidemargin=0in      %
%\textwidth=6.5in        %
%\textheight=9.0in       %
%\headsep=0.25in         %

% Setup the header and footer
\pagestyle{fancy}
% This is used to trace down (pin point) problems
% in latexing a document:
%\tracingall

% Includes a figure
% The first parameter is the label, which is also the name of the figure
% with or without the extension (e.g., .eps, .fig, .png, .gif, etc.)
% IF NO EXTENSION IS GIVEN, LaTeX will look for the most appropriate one.
% This means that if a DVI (or PS) is being produced, it will look for
% an eps. If a PDF is being produced, it will look for nearly anything
% else (gif, jpg, png, et cetera). Because of this, when I generate figures
% I typically generate an eps and a png to allow me the most flexibility
% when rendering my document.
% The second parameter is the width of the figure normalized to column width
% (e.g. 0.5 for half a column, 0.75 for 75% of the column)
% The third parameter is the caption.
\newcommand{\scalefig}[3]{
  \begin{figure}[ht!]
    % Requires \usepackage{graphicx}
    \centering
    \includegraphics[width=#2\columnwidth]{#1}
    %%% I think \captionwidth (see above) can go away as long as
    %%% \centering is above
    % \captionwidth{#2\columnwidth}%
    \caption{#3}
    \label{#1}
  \end{figure}}

% Includes a MATLAB script.
% The first parameter is the label, which also is the name of the script
% without the .m.
% The second parameter is the optional caption.
\newcommand{\matlabscript}[2]
{\begin{itemize}\item[]\lstinputlisting[caption=#2,label=#1]{#1.m}\end{itemize}}
\title{Projection Operators and Canonical Symmetry for the Pseudo Plane Wave Modes}
\date{\today}
\author{Nathan Hitchings}
\begin{document}
\maketitle
Let $\Phi(t,x)$ be a solution to the Klein-Gordon equation and assume
we can expand $\Phi$ as follows:
\begin{equation}
  \Phi(t,x)=\int_kb(k)V_k+\bar{b}(k)\bar{V}_k\, dk
\end{equation}
where 
\begin{align}
  V_k&=\frac{1}{2\sqrt{\pi}}e^{ikx}\left(\cos(\omega_{k}t)-\frac{i}{\omega_{k}}\sin(\omega_{k}t)\right)\\
  \bar{V}_k&=\frac{1}{2\sqrt{\pi}}e^{-ikx}\left(\cos(\omega_{k}t)+\frac{i}{\omega_{k}}\sin(\omega_{k}t)\right)
\end{align}
and 
\begin{align}
  b(k)&=\frac{1}{\sqrt{2}}\left(\hat{\phi}(k)+i\hat{\pi}(k)\right)\\
  \bar{b}(k)&=\frac{1}{\sqrt{2}}\left(\hat{\phi}(-k)-i\hat{\pi}(-k)\right).
\end{align}

First, we find the positive projection operator, $P^+$. This operator is defined as
\begin{equation}\label{eq:pplus}
  P^+\Phi=\int_kb(k)V_k\, dk.
\end{equation}
Substituting the expressions for $b(k)$ and $V_k$ into
\eqref{eq:pplus} we get 
\begin{equation}
  P^+\Phi=\int_k\frac{1}{\sqrt{2}}\left(\hat{\phi}(k)+i\hat{\pi}(k)\right)\frac{1}{2\sqrt{\pi}}e^{ikx}\left(\cos(\omega_{k}t)-\frac{i}{\omega_{k}}\sin(\omega_{k}t)\right)\,dk  
\end{equation}
Set $f(x)=(P^+\Phi)(0,x)$. Then, 
\begin{align*}
  f(x)&=\int_{k}\frac{1}{\sqrt{2}}\left(\hat{\phi}(k)+i\hat{\pi}(k)\right)\frac{1}{2\sqrt{\pi}}e^{ikx}\, dk\\
  &=\frac{1}{2}\left(\frac{1}{\sqrt{2\pi}}\int_{k}\hat{\phi}(k)e^{ikx}\, dk+\frac{1}{\sqrt{2\pi}}\int_{k}\hat{\pi}(k)e^{ikx}\, dk\right)\\
  &=\frac{1}{2}(\phi(x)+i\pi(x))
\end{align*}
Now set $g(x)=(P^+\dot{\Phi})(0,x)$. Then 
\begin{align*}
  g(x)&=-\int_{k}\frac{1}{\sqrt{2}}\left(\hat{\phi}(k)+i\hat{\pi}(k)\right)\frac{i}{2\sqrt{\pi}}e^{ikx}\, dk\\
  &=\frac{1}{2}\left(-\frac{i}{\sqrt{2\pi}}\int_{k}\hat{\phi}(k)e^{ikx}\, dk+\frac{1}{\sqrt{2\pi}}\int_{k}\hat{\pi}(k)e^{ikx}\, dk\right)\\
  &=\frac{1}{2}(\pi(x)-i\phi(x))
\end{align*}
Hence,
\begin{equation}
  P^+\begin{pmatrix}\phi\\ \pi\end{pmatrix}=\begin{pmatrix}f\\
    g\end{pmatrix}=\frac{1}{2}\begin{pmatrix}\phi(x)+i\pi(x)\\ \pi(x)-i\phi(x)\end{pmatrix}
\end{equation}

Next we need to find the negative projection operator, $P^-$. This operator is
defined as 
\begin{align}
P^-\Phi&=\int_k\bar{b}(k)\bar{V}_k\, dk\\
&=\int_{k}\frac{1}{\sqrt{2}}\left(\hat{\phi}(-k)-i\hat{\pi}(-k)\right)\frac{1}{2\sqrt{\pi}}e^{-ikx}\left(\cos(\omega_{k}t)
+\frac{i}{\omega_{k}}\sin(\omega_{k}t)\right)\, dk
\end{align}
Now, let $f(x)=P^{-}\Phi(0,x)$. Then
\begin{equation}
f(x)=\frac{1}{2}\left(\frac{1}{\sqrt{2\pi}}\int_{-\infty}^{\infty}\hat{\phi}(-k)e^{-ikx}\, dk-\frac{i}{\sqrt{2\pi}}\int_{-\infty}^{\infty}\hat{\pi}(-k)e^{-ikx}\, dk\right)
\end{equation}
Let $z=-k$, then
\begin{align*}
f(x)&=\frac{1}{2}\left(-\frac{1}{\sqrt{2\pi}}\int_{\infty}^{-\infty}\hat{\phi}(z)e^{izx}\, dz+\frac{i}{\sqrt{2\pi}}\int_{\infty}^{-\infty}\hat{\pi}(z)e^{izx}\, dz\right)\\
&=\frac{1}{2}\left(\frac{1}{\sqrt{2\pi}}\int_{-\infty}^{\infty}\hat{\phi}(z)e^{izx}\, dz-\frac{i}{\sqrt{2\pi}}\int_{-\infty}^{\infty}\hat{\pi}(z)e^{izx}\, dz\right)\\
&=\frac{1}{2}(\phi(x)-i\pi(x))
\end{align*}
Next, let $g(x)=(P^{-}\dot{\Phi})(0,x)$. Then,
\begin{equation}
g(x)=\frac{1}{2}\left(\frac{i}{\sqrt{2\pi}}\int^{\infty}_{-\infty}\hat{\phi}(-k)e^{-ikx}\, dk+\frac{1}{\sqrt{2\pi}}\int^{\infty}_{-\infty}\hat{\pi}(-k)e^{-ikx}\, dk\right)
\end{equation}
Again, let $z=-k$. Then,
\begin{align*}
g(x)&=\frac{1}{2}\left(-\frac{i}{\sqrt{2\pi}}\int^{-\infty}_{\infty}\hat{\phi}(z)e^{izx}\, dz-\frac{1}{\sqrt{2\pi}}\int^{-\infty}_{\infty}\hat{\pi}(z)e^{izx}\, dz\right) \\
&=\frac{1}{2}\left(\frac{i}{\sqrt{2\pi}}\int^{\infty}_{-\infty}\hat{\phi}(z)e^{izx}\, dz+\frac{1}{\sqrt{2\pi}}\int^{\infty}_{-\infty}\hat{\pi}(z)e^{izx}\, dz\right) \\
&=\frac{1}{2}(\pi(x)+i\phi(x))
\end{align*}
Therefore,
\begin{equation}
P^{-}\begin{pmatrix}\phi \\ \pi\end{pmatrix}=\begin{pmatrix}f \\ g\end{pmatrix}=\frac{1}{2}\begin{pmatrix}\phi(x)-i\pi(x)\\ \pi(x)+i\phi(x)\end{pmatrix}
\end{equation}
We can see plainly now that, with the pseudo-plane wave modes,
\begin{equation}
P^-\begin{pmatrix}\phi\\ \pi\end{pmatrix}=\overline{P^+\begin{pmatrix}\phi\\ \pi\end{pmatrix}}
\end{equation}
The canonical symmetry $J$ is defined as $J=P^{+}-P^{-}$. Thus,
\begin{align*}
J\begin{pmatrix}\phi \\ \pi\end{pmatrix}&=(P^{+}-P^{-})\begin{pmatrix}\phi \\ \pi\end{pmatrix}=P^+\begin{pmatrix}\phi \\ \pi\end{pmatrix}-P^-\begin{pmatrix}\phi \\ \pi\end{pmatrix}\\
&=\begin{pmatrix}i\phi \\ -i\pi\end{pmatrix}=i\begin{pmatrix}\phi \\ -\pi\end{pmatrix}
\end{align*}

\begin{claim}
For $P^{\pm}$ from above we have
\begin{equation}
P^{\pm}\left(P^{\pm}\begin{pmatrix}\phi \\ \pi\end{pmatrix}\right)=P^{\pm}\begin{pmatrix}\phi \\ \pi\end{pmatrix}
\end{equation}
\end{claim}
\begin{proof}[Reason]
Let us consider the operator $P^{+}$. Let \begin{equation}
\begin{pmatrix}f\\ g\end{pmatrix}=P^{+}\begin{pmatrix}\phi\\ \pi\end{pmatrix}
\end{equation}
Then \begin{align*}
P^{+}\left(P^{+}\begin{pmatrix}\phi \\ \pi\end{pmatrix}\right)&=P^{+}\begin{pmatrix}f\\ g\end{pmatrix}=\frac{1}{2}\begin{pmatrix}f(x)+ig(x)\\ g(x)-if(x)\end{pmatrix}\\
&=\frac{1}{2}\begin{pmatrix}\frac{1}{2}(\phi(x)+i\pi(x))+i(\frac{1}{2}(\pi(x)-i\phi(x)))\\ \frac{1}{2}(\pi(x)-i\phi(x))-i(\frac{1}{2}(\phi(x)+i\pi(x)))\end{pmatrix}\\
&=\frac{1}{2}\begin{pmatrix}\phi(x)+i\pi(x)\\ \pi(x)-i\phi(x)\end{pmatrix}\\
&=P^{+}\begin{pmatrix}
\phi \\ \pi
\end{pmatrix}
\end{align*}
as required. A nearly identical argument can be used to show the
result for $P^{-}$.
\end{proof}
\end{document}











